\documentclass[11pt, addpoints, answers]{exam}

\usepackage[utf8]{inputenc}
\usepackage[T1]{fontenc}
\usepackage[margin  = 1in]{geometry}
\usepackage{amsmath, amscd, amssymb, amsthm, verbatim}
\usepackage{mathabx}
\usepackage{setspace}
\usepackage{float}
\usepackage{color}
\usepackage{graphicx}   
\usepackage[colorlinks=true]{hyperref}
\usepackage{tikz}

\usetikzlibrary{shapes,arrows}
%%%<
\usepackage{verbatim}
%%%>
\usetikzlibrary{automata,arrows,positioning,calc}

\usetikzlibrary{trees}

\shadedsolutions
\definecolor{SolutionColor}{RGB}{214,240,234}

\newcommand{\bbC}{{\mathbb C}}
\newcommand{\R}{\mathbb{R}}            % real numbers
\newcommand{\bbR}{{\mathbb R}}
\newcommand{\Z}{\mathbb{Z}}            % integers
\newcommand{\bbZ}{{\mathbb Z}}
\newcommand{\bx}{\mathbf x}            % boldface x
\newcommand{\by}{\mathbf y}            % boldface y
\newcommand{\bz}{\mathbf z}            % boldface z
\newcommand{\bn}{\mathbf n}            % boldface n
\newcommand{\br}{\mathbf r}            % boldface r
\newcommand{\bc}{\mathbf c}            % boldface c
\newcommand{\be}{\mathbf e}            % boldface e
\newcommand{\bE}{\mathbb E}            % blackboard E
\newcommand{\bP}{\mathbb P}            % blackboard P

\newcommand{\ve}{\varepsilon}          % varepsilon
\newcommand{\avg}[1]{\left< #1 \right>} % for average
%\renewcommand{\vec}[1]{\mathbf{#1}} % bold vectors
\newcommand{\grad}{\nabla }
\newcommand{\lb}{\langle }
\newcommand{\rb}{\rangle }

\def\Bin{\operatorname{Bin}}
\def\Var{\operatorname{Var}}
\def\Geom{\operatorname{Geom}}
\def\Pois{\operatorname{Pois}}
\def\Exp{\operatorname{Exp}}
\newcommand{\Ber}{\operatorname{Ber}}
\def\Unif{\operatorname{Unif}}
\def\No{\operatorname{N}}
\newcommand{\E}{\mathbb E}            % blackboard E
\def\th{\theta }            % theta shortcut
\def\V{\operatorname{Var}}
\def\Var{\operatorname{Var}}
\def\Cov{\operatorname{Cov}}
\def\Corr{\operatorname{Corr}}
\newcommand{\epsi}{\varepsilon}            % epsilon shortcut

\providecommand{\norm}[1]{\left\lVert#1\right\rVert} %norm
\providecommand{\abs}[1]{\left \lvert#1\right \rvert} %absolute value

\DeclareMathOperator{\lcm}{lcm}
\newcommand{\ds}{\displaystyle}	% displaystyle shortcut

% Distributions.
\newcommand*{\UnifDist}{\mathsf{Unif}}
\newcommand*{\ExpDist}{\mathsf{Exp}}
\newcommand*{\DepExpDist}{\mathsf{DepExp}}
\newcommand*{\GammaDist}{\mathsf{Gamma}}
\newcommand*{\LognormalDist}{\mathsf{LogNorm}}
\newcommand*{\WeibullDist}{\mathsf{Weib}}
\newcommand*{\ParetoDist}{\mathsf{Par}}
\newcommand*{\NormalDist}{\mathsf{Normal}}

\newcommand*{\GeometricDist}{\mathsf{Geom}}
\newcommand*{\NegBinomialDist}{\mathsf{NegBin}}
\newcommand*{\BinomialDist}{\mathsf{Bin}}
\newcommand*{\PoissonDist}{\mathsf{Poisson}}
\newcommand*{\Prob}{\mathbb{P}}
% \newcommand*{\Cov}{\mathsf{Cov}}


\def\semester{2023-2024}
\def\course{Calcul Stochastique Appliqué}
\def\title{\MakeUppercase{Examen final}}
\def\name{Pierre-O Goffard}
%\def\name{Professor Wildman}

\setlength\parindent{0pt}

\cellwidth{.35in} %sets the minimum width of the blank cells to length
\gradetablestretch{2.5}

%\bracketedpoints
%\pointsinmargin
%\pointsinrightmargin

\begin{document}


\runningheader{\course  \vspace*{.25in}}{}{\title \vspace*{.25in}}
%\runningheadrule
\runningfooter{}{Page \thepage\ of \numpages}{}

% \firstpageheader{Name:\enspace\hbox to 2.5in{\hrulefill}\\  \vspace*{2em} Section: (circle one) TR: 3-3:50 \textbar\, TR: 5-5:50 \textbar\,  TR: 6-6:50(Xu) \textbar\,  TR: 6-6:50 }{}{Perm \#: \enspace\hbox to 1.5in{\hrulefill}\\ \vspace*{2em} Score:\enspace\hbox to .6in{\hrulefill} $/$\numpoints}
\extraheadheight{.25in}

\hrulefill

\vspace*{1em}

% Heading
{\center \textsc{\Large\title}\\
	\vspace*{1em}
	\course -- \semester\\
	Pierre-O Goffard\\
}
\vspace*{1em}

\hrulefill

\vspace*{2em}

\noindent {\bf\em Instructions:} On éteint et on range son téléphone.
\begin{itemize}
	\item La calculatrice et les appareils éléctroniques ne sont pas autorisés.
	\item Vous devez justifier vos réponses de manière claire et concise.
	\item Vous devez écrire de la manière la plus lisible possible. Souligner ou encadrer votre réponse finale.
	\item \underline{Document autorisé:} Une feuille manuscrite recto-verso

\end{itemize}


\begin{center}
	\gradetable[h]
\end{center}

\smallskip

\begin{questions}
\question[2] Soit $(B_t)_{t\geq 0}$ un mouvement Brownien. Montrer que le processus défini par 
$$
X_t =  \frac{B_{c^2t}}{c}, \text{ }t\geq 0,
$$
pour $c>0$ est un mouvement brownien.
\begin{solution}
On montre qu'il s'agit d'un processus gaussien. Soient $t_1<t_2$ deux instants et $a_1, a_2\in \R$ alors 
\begin{eqnarray*}
a_1X_{t_1}+ a_2X_{t_2} &=&  a_1\frac{B_{c^2t_1}}{c}+ a_2\frac{B_{c^2t_2}}{c}\\
&=&\frac{(a_1+a_2)}{c}B_{c^2t_1}+ \frac{a_2}{c}(B_{c^2t_2}-B_{c^2t_1})\\
\end{eqnarray*}
est une va gaussienne comme somme de va gaussiennes indépendantes. On note que 
$$
\E(X_t) = 0
$$
et 
$$
C(s,t) = Cov(X_s, X_t) = \frac{Cov(B_{c^2s}, B_{c^2t})}{c^2} = \frac{c^2s\land c^2t}{c^2} = s\land t.
$$
$X_t$ est un processus gaussien dont la fonction de moyenne et de covariance sont identiques à celles du mouvement browninen. Il s'agit donc bien d'un mouvement brownien. 
\end{solution}

\question Soit $N = (N_t)_{t\geq 0}$ un processus de Poisson d'intensité $\lambda$ et $\mathcal{F}_t$ sa filtration. Soit 
$$
S_t = S_0\exp(\mu t - bN_t),\text{ } t\geq 0, 
$$
où $\mu, b>0$, la valeur d'un actif financier risqué. On suppose qu'il existe également un actif sans risque telle que 
$$
S_t^0 = S_0^0e^{rt},\text{ }t\geq0.
$$
On supposera que $S_0 = S_0^0 = 1$.
\begin{parts}
\part[1] Calculer $E(S_t)$
\begin{solution}
$$
E(S_t) = S_0e^{\mu t }\E(e^{-bN_t}) = S_0\exp(\mu t + \lambda t (e^{-b} - 1) )
$$
\end{solution}
\part[1] On note 
$$
\tilde{S}_t = e^{-rt}S_t,\text{ } t\geq 0,
$$
la valeur actualisée de l'actif au taux sans risque. Déterminer la valeur $\lambda^{\ast}$ de $\lambda$ pour que $(\tilde{S}_t)_{t\geq 0}$ soit une martingale. On exprimera $\lambda^\ast$ en fonction de $\mu, r$, et $b$.
\begin{solution}
Il faut que 
$$
\lambda^\ast = \frac{\mu - r}{1 - e^{-b}}. 
$$
\end{solution}
\part[2] La mesure de probabilité $\mathbb{Q}$ sous laquelle $(N_t)_{t\geq 0}$ est un procesus de Poisson d'intensité $\lambda^\ast$ correspond à la probabilité risque-neutre. Donner le prix 
$$
\E_\mathbb{Q}\left[e^{-rT}(S_T - K)_+\right],
$$
où $(\cdot)_+ = \max(\cdot, 0$ est la partie positive, d'un call européen de maturité $T$ et de prix d'exercice $K$ sous $\mathbb{Q}$, en fonction de $r$, $\lambda^\ast$, $\mu$, $K$, $T$, et de 
$$
F(x;\lambda) = \sum_{k=0}^x\frac{\lambda^k e^{-\lambda}}{k!},\text{ }x\in \mathbb{N},
$$
la fonction de répartition d'une variable aléatoire de loi de Poisson de paramètre $\lambda>0$.
\begin{solution}
On a 

$$\E_\mathbb{Q}\left[e^{-rT}(S_T - K)_+\right] = e^{-rT}S_0F(d;\lambda^\ast e^{-b}T) - KF(d,\lambda^\ast T),$$
où 
$$
d = \left\lfloor \frac{\mu T -\ln(K/S_0)}{b}\right \rfloor.
$$
\end{solution}
\end{parts}
\question Soit $(B_t)_{t\geq 0}$ un mouvement Brownien et $(X_t)_{t\geq 1}$ un processus d'Itô dont la dynamique est donnée par 
$$
\text{d}X_t = \left(te^{-t} - \frac{t^{-\frac12}}{2}+X_t t^{-1}\right)\text{d}t+2t\text{d}B_t,
$$
et $X_1 = 1$.
\begin{parts}
\part[1] On définit le processus $Y$ par
$$
Y_t= \frac{X_t}{t}- t^{-\frac 12},\text{ }t\geq 1.
$$
Trouver l'équation différentielle stochastique vérifiée par le processus $Y$. 
\begin{solution}
On applique le lemme d'Ito au processus $Y_t = f(X_t, t)$. On a 
$$
\frac{\partial f}{\partial t} = -\frac{x}{t^2} + \frac{t^{-\frac 32}}{2}\text{, }\frac{\partial f}{\partial x} = \frac 1t\text{, et }\frac{\partial^2 f}{\partial x^2} = 0.
$$
On en déduit que 
$$
\text{d}Y_t = e^{-t}\text{d}t + 2\text{d}B_t.
$$
\end{solution}
\part[1] Quel est la loi de $Y_t$ pour $t> 1$?
\begin{solution}
En intégrant l'équation de $Y$ entre $1$ et $t$, il vient 
$$
Y_t = e^{-1} - e^{-t} + 2(B_t - B_1)
$$
On en déduit que 
$$
Y_{t}\sim \text{Normal}(e^{-1} - e^{-t}, 4(t -1))
$$
\end{solution}
\part[1] En déduire la loi $X_t$ pour $t> 1$.
\begin{solution}
On a 
$$
X_t = tY_t + \sqrt{t}
$$
On en déduit que 
$$
X_t\sim\text{Normal}(t(e^{-1} - e^{-t}) + \sqrt{t}, 4t^2(t-1))
$$
\end{solution}
\end{parts}
\question Soient $B = (B_t)_{t\geq0}$ un mouvement Brownien et $(\mathcal{F}_t)_{t\geq 0}$ sa filtration. 
\begin{parts}
\part[1] Montrer que $B$ est une martingale. (on admettra que $\E(|B_t|) <\infty$ pour tout $t\geq0$)
\begin{solution}
Voir le cours
\end{solution}
\part[2] Soient $a>0$ et $b>0$. On définit les temps d'arrêt
$$
\tau_a^+ = \inf\{t\geq 0 \text{ ; }B_t = a\},\text{ }\tau_b^- = \inf\{t\geq 0 \text{ ; }B_t = -b\},\text{ et }\tau = \tau_a^+\land\tau_b^-.
$$
On suppose qu'ils sont finis presque sûrement. Montrer que 
$$
\Prob(\tau = \tau_a^+) = \frac{b}{a+b}.
$$
\underline{Indication:} On utilise le théorème du temps d'arrêt sur $B$.
\begin{solution}
Par application du théorème du temps d'arrêt au temps $\tau$, il vient
\begin{eqnarray*}
0 = \E(B_0)&=&\E(B_\tau)\\
&=& a \Prob(\tau = \tau_a^+) -b(1- \Prob(\tau = \tau_a^+))\\
&=&(a+b)\Prob(\tau = \tau_a^+) - b
\end{eqnarray*}
On en déduit que 
$$
\Prob(\tau = \tau_a^+) = \frac{b}{a+b}.
$$
\end{solution}
\part[1] On définit 
$$
M_t = \int_0^t B_u\text{d}u - \frac 13 B_t^3.
$$
Calculer $\E(M_t)$ pour tout $t\geq 0$
\begin{solution}
$0$
\end{solution}
\part[1] Montrer que $(M_t)_{t\geq 0}$ est une martingale. (on admettra que $\E(|M_t|) <\infty$, pour tout $t\geq0$)\\
\underline{Indication:} On a pour $s<t$
$$
\E\left(\int_s^t B_u\text{d}u\Big\rvert\mathcal{F}_s\right) =\int_s^t \E\left(B_u\rvert\mathcal{F}_s\right)\text{d}u  
= \int_s^t B_s\text{d}u = B_s\cdot (t-s)
$$
\begin{solution}
Soit $s<t$, on a 
\begin{eqnarray*}
\E(M_t|\mathcal{F}_s) &=& \E\left(\int_0^t B_u\text{d}u - B_t^3\Big\rvert \mathcal{F}_s\right)\\
&=& \E\left(\int_0^s B_u\text{d}u + \int_s^t B_u\text{d}u - (B_t-B_s+B_s)^3\Big\rvert \mathcal{F}_s\right)\\
&=& \int_0^s B_u\text{d}u + B_s\cdot (t-s) -\frac 13\left(\E(B_t-B_s)^3 + B_s^3 + 3B_s(t-s)\right)\\
&=&M_s.
\end{eqnarray*}
\end{solution}
\part[1] Déduire des questions précédentes (b et d en particulier) que l'aire sous $(B_t)_{t\geq 0}$ jusqu'à $\tau$ est en moyenne égale à 
$$
\frac{ab(a-b)}{3}.
$$
\\

\underline{Indication:} On utilise le théorème du temps d'arrêt optionnel au temps $\tau$ sur le processus $(M_t)_{t\geq 0}$.
\begin{solution}
\begin{eqnarray*}
0=\E(M_0) &=&\E(M_\tau) =\E\left(\int_{0}^\tau B_u\text{d}u\right) -\frac{1}{3}\E(B_\tau^3) \\
\end{eqnarray*}
On déduit que 
$$
\E\left(\int_{0}^\tau B_u\text{d}u\right) = \frac{1}{3}\left(a^3\frac{b}{a+b}-b^3\frac{a}{a+b}\right)= ab(a-b)/3
$$
\end{solution}
\end{parts}

\end{questions}
%-------------------------------TABLE-------------------------------
\newpage
\hrule
\vspace*{.15in}
\begin{center}
  \large\MakeUppercase{Formulaire}
\end{center}
\vspace*{.15in}
\hrule
\vspace*{.25in}

\renewcommand\arraystretch{3.5}
\begin{table}[H]
\begin{center}
\footnotesize
\begin{tabular}{|c|c|c|c|c|c|}

\hline
Nom & abbrev. & Loi & $\E(X)$ & $\Var(X)$ & $\E\left(e^{tX}\right)$\\
\hline\hline
Binomial & $\Bin(n,p)$ & $\binom{n}{k}p^k(1-p)^{n-k},\text{ }k = 1,\ldots, n$ & $np$ & $np(1-p)$ & $[(1-p)+pe^t]^n$\\
\hline
Poisson & $\Pois(\lambda)$ & $e^{-\lambda}\dfrac{\lambda^k}{k!},\text{ }k\geq 0$ & $\lambda$ & $\lambda$ &$ \exp(\lambda(e^t-1))$\\
\hline
Geometric & $\Geom(p)$ & $(1-p)^{k-1}p,\text{ }k\geq 1$ & $\dfrac{1}{p}$ & $\dfrac{1-p}{p^2}$ & $\frac{pe^t}{1-(1-p)e^t}$ pour  $t<-\ln(1-p)$\\
\hline
Uniform & $\Unif(a,b)$ & $\begin{cases} \dfrac{1}{b-a} & a\leq t\leq b\\ 0 & \text{sinon}\end{cases}
$ & $\dfrac{a+b}{2}$ & $\dfrac{(b-a)^2}{12}$ & $\frac{e^{tb}-e^{ta}}{t(b-a)}$\\
\hline
Exponential & $\Exp(\lambda)$ & $\begin{cases} \lambda e^{-\lambda t} & t\geq 0 \\ 0 & t<0\end{cases}$ & $\dfrac{1}{\lambda}$ & $\dfrac{1}{\lambda^2}$ & $\frac{\lambda}{\lambda -t}$ pour $t<\lambda$\\
\hline
Normal & $\No(\mu,\sigma^2)$ & $\left(\dfrac{1}{\sqrt{2\pi\sigma^2}}\right)\operatorname{exp}{\left(\dfrac{-(t-\mu)^2}{2\sigma^2}\right)}$ & $\mu$ & $\sigma^2$ & $e^{\mu t}e^{\sigma^2t^2/2}$\\
\hline
\end{tabular}
\end{center}
\end{table}%

\end{document}