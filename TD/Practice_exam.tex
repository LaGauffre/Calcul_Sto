\documentclass[11pt, answers]{exam}
\usepackage[utf8]{inputenc}
\usepackage[T1]{fontenc}
\usepackage{amsmath, amssymb, amsopn, color, tikz, mathtools}
\usepackage[margin=1in]{geometry}
\usepackage{titlesec}
\usepackage{tipa}
\usepackage{hyperref}


% Style
\setlength\parindent{0pt}
\shadedsolutions

% Define course info
\def\semester{Semestre 2}
\def\course{Calcul stochastique M1 DUAS}
\def\name{P.-O. Goffard}
%\def\quizdate{10/5, 10/6}
\def\hwknum{}
%\def\title{\MakeUppercase{Homework \hwknum -- quiz \quizdate }}
\def\title{\MakeUppercase{Entrainement Examen}}
% Distributions.
\newcommand*{\UnifDist}{\mathsf{Unif}}
\newcommand*{\ExpDist}{\mathsf{Exp}}
\newcommand*{\DepExpDist}{\mathsf{DepExp}}
\newcommand*{\GammaDist}{\mathsf{Gamma}}
\newcommand*{\LognormalDist}{\mathsf{LogNorm}}
\newcommand*{\WeibullDist}{\mathsf{Weib}}
\newcommand*{\ParetoDist}{\mathsf{Par}}
\newcommand*{\NormalDist}{\mathsf{Normal}}

\newcommand*{\GeometricDist}{\mathsf{Geom}}
\newcommand*{\NegBinomialDist}{\mathsf{NegBin}}
\newcommand*{\BinomialDist}{\mathsf{Bin}}
\newcommand*{\PoissonDist}{\mathsf{Poisson}}
\newcommand*{\Cov}{\mathsf{Cov}}

% Sets of numbers.
\newcommand*{\RL}{\mathbb{R}}
\newcommand*{\N}{\mathbb{N}}
\newcommand*{\NZ}{\mathbb{N}_0}
% \newcommand*{\NL}{\mathbb{N}_+}

\newcommand*{\cond}{\mid}
\newcommand*{\given}{\,;\,}

%Probability symbols
\newcommand*{\Prob}{\mathbb{P}}
\newcommand*{\Q}{\mathbb{Q}}
\newcommand*{\E}{\mathbb{E}}
\newcommand*{\F}{\mathcal{F}}
% Regarding spacing and abbreviations.
\usepackage{xspace}

% Acronyms
% \@\xspace doesn't add space if next char is punctuation
% However, these will give 2 .'s if used at end of sentence.
\newcommand*{\eg}{e.g.\@\xspace}
\newcommand*{\ps}{p.s.\@\xspace}
\newcommand*{\ie}{i.e.\@\xspace}
\newcommand*{\va}{v.a.\@\xspace}
\newcommand*{\iid}{i.i.d.\@\xspace}
\newcommand*{\ssi}{s.s.i.\@\xspace}
\newcommand*{\cf}{c.f.\@\xspace}
\newcommand*{\pdf}{p.d.f.\@\xspace}
\newcommand*{\pmf}{p.m.f.\@\xspace}
\newcommand*{\cdf}{c.d.f.\@\xspace}
\newcommand*{\SMC}{\textbf{SMC}\@\xspace}
\newcommand*{\MCMC}{\textbf{MCMC}\@\xspace}
\newcommand*{\VF}{\textbf{VF}\@\xspace}


\newcommand*{\iidSim}{\overset{\mathrm{i.i.d.}}{\sim}}
\newcommand*{\bt}{\bm{\theta}}
\newcommand*{\bTheta}{\bm{\Theta}}
\newcommand*{\bbeta}{\bm{\beta}}
\newcommand*{\bx}{\mathbf{x}}
\newcommand*{\bs}{\bm{s}}
\newcommand*{\bu}{\bm{u}}
\newcommand*{\bn}{\bm{n}}

% Roman versions of things.
\newcommand*{\dd}{\mathop{}\!\mathrm{d}}
\newcommand*{\e}{\mathrm{e}}
\DeclareMathOperator*{\argmax}{arg\,max}
\DeclareMathOperator*{\argmin}{arg\,min}

\newcommand*{\norm}[1]{\lVert{} #1\rVert}

% \DeclarePairedDelimiterXPP{\ind}[1]{\ind_}{\{}{\}}{}{#1}
\newcommand*{\ind}{\mathbb{I}}
\def\euro{\mbox{\raisebox{.25ex}{{\it =}}\hspace{-.5em}{\sf C}}}
  \everymath{\displaystyle}
% \newcommand{\limsup}{\overline{\lim}\,}            % blackboard P
% \newcommand{\liminf}{\underline{\lim}\,}            % blackboard P

\begin{document}

% Heading
{\center \textsc{\Large\title}\\
	\vspace*{1em}
	\course -- \semester\\
	\name\\
	\vspace*{2em}
	\hrule
\vspace*{2em}}
\begin{questions}
\question Soit $S = (S_n)_{n\geq 0}$ un processus défini par 
$$
S_0 = 0, \text{ et }S_n = \sum_{i=1}^n\xi_i,\text{ }n\geq 1,
$$
où $(\xi_i)_{i\geq 1}$ ets une suite de \va \iid telles que 
$$
\Prob(\xi_i = -1) = \Prob(\xi_i = 1) = 1/2.
$$
Soit $\F_n = \sigma(\xi_i,i\leq n)$, la filtration du processus $S$.
\begin{parts}
\part Montrer que $S$ est une $\F_n-$martingale.
\begin{solution}
Vu en cours
\end{solution}
\part Montrer que 
$$
Y_n = S_n^2 - n,\text{ }n\geq 1.
$$ 
est une $\F_n-$martingale.
\begin{solution}
\begin{eqnarray*}
\E(Y_{n+1}|\F_n)&=&\E(S_{n+1}^2 - n - 1|\F_n)\\
&=&\E[(S_{n}+\xi_{n+1})^2|\F_n) - n - 1\\
&=&\E[S_{n}^2+2S_n\xi_{n+1}+ \xi_{n+1}^2|\F_n) - n - 1\\
&=&S_{n}^2+2S_n\E(\xi_{n+1})+ \E(\xi_{n+1}^2) - n - 1\\
&=&S_{n}^2+0+ 1 - n - 1\\
&=&S_{n}^2- n  = Y_n\\
\end{eqnarray*}
\end{solution}
\part Soient $a,b\in \N$ tels que $a,b>0$, on définit
$$
\tau_a^+ = \inf\{n\geq 0\text{ ;  }S_n = a\},\text{ }\tau_a^- = \inf\{n\geq 0\text{ ;  }S_n = -b\}, \text{ et }\tau = \tau_a^+\land \tau_b^-.
$$
Montrer que $\tau_a^+,\tau_b^-$ et $\tau$ sont des temps d'arrêts.
\begin{solution}
On a 
$$
\{\tau_a^+ = n\} = \bigcap_{k = 0}^{n-1}\{S_k < a \}\cap\{S_n = a \}\in \F_n
$$
et
$$
\{\tau_b^- = n\} = \bigcap_{k = 0}^{n-1}\{S_k > -b \}\cap\{S_n = b \}\in \F_n.
$$
On en déduit que $\tau_a^+$ et $\tau_b^-$ sont des temps d'arrêt. $\tau$ est un temps d'arrêt en tant que minimum de temps d'arrêt, en effet le minimum de deux applications mesuarbales est une application mesurable.
\end{solution}
\part Calculer la probabilité
$$
\Prob(\tau = \tau_a^+).
$$
\underline{Indication:} Il faut utiliser le théorème de temps d'arrêt.

\begin{solution}
On note d'abord que 
$$
\Prob(\tau = \tau_a^+) = 1 - \Prob(\tau = \tau_b^-).
$$
On applique le théorème du temps d'arrêt au temps $\tau$, il vient 
$$
\E(S_\tau) = \E(S_0) = 0.
$$
D'autre part, on a 
\begin{eqnarray*}
\E(S_\tau) &=& \E(S_{\tau_a^+}\ind_{\tau  = \tau_a^+}+S_{\tau_b^-}\ind_{\tau  = \tau_b^-})\\
 &=& a\Prob(\tau  = \tau_a^+)-b\Prob(\tau  = \tau_b^-)\\
 &=& (a+b)\Prob(\tau  = \tau_a^+)-b
\end{eqnarray*}
On en déduit que 
$$
\Prob(\tau = \tau_a^+) = \frac{b}{a+b}.
$$
\end{solution}
\part Calculer $\E(\tau)$.\\
\underline{Indication:} Il faut utiliser le théorème de temps d'arrêt et le processus $Y$.
\begin{solution}
On applique le théorème du temps d'arrêt en $\tau$ sur le processus $Y$. Il vient 
$$
\E(Y_\tau) = \E(Y_0) = 0,
$$
d'une part. D'autre part, on a 
\begin{eqnarray*}
\E(Y_\tau) &=& \E(S_\tau^2) - \E(\tau)\\
\end{eqnarray*}
On en déduit que $\E(S_\tau^2) = \E(\tau)$, or 
\begin{eqnarray*}
\E(S_\tau^2)&=&\E(S_{\tau_a^+}^2\ind_{\tau  = \tau_a^+}+S_{\tau_b^-}^2\ind_{\tau  = \tau_b^-})\\
&=&a^2\Prob(\tau  = \tau_a^+)+b^2\Prob(\tau  = \tau_b^-)\\
&=&a^2\frac{b}{a+b}+b^2\frac{a}{a+b} = ab
\end{eqnarray*}
\end{solution}
\end{parts}
\question Soit $Y\sim\NormalDist(0,1)$, le processus 
$$
X_t = \sqrt{t}Y,
$$

est-il un mouvement brownien standard?
\begin{solution}
Non, par exemple 
$$
X_t-X_s\sim\NormalDist(0, t+s - \sqrt{ts}).
$$
\end{solution}
\question Soit $B=(B_t)_{t\geq 0}$ un mouvement brownien et $(\F_t)_{t\geq 0}$ sa filtration. 
\begin{parts}
\part Le processus 
$$
X_t = \frac{1}{\sqrt{a}}B_{at},\text{ }t\geq 0\text{ pour }a>0,
$$
est il un mouvement brownien?
\begin{solution}
On a 
\begin{enumerate}
	\item $X_0 = 0$
	\item $X_t$ et $X_{s+t} - X_s$ sont indépendants pour tout $s,t>0$
	\item $X_{s+t} - X_s\NormalDist(0,t)$ pour tout $s,t>0$
	\item $t\mapsto X_t$ est continue par opération sur les fonctions continues.
\end{enumerate}
On peut aussi vérifier que $X_t$ est un processus gaussien avec 
$$
b_1X_{t_1}+b_2X_{t_2} = \left(\frac{b_1}{\sqrt{a}}+\frac{b_2}{\sqrt{a}}\right)B_{at_1} + \frac{b_2}{\sqrt{a}}(B_{at_2} - B_{at_1}),\text{ pour }t_1<t_2\text{ et }b_1,b_2\in \RL.
$$ 
de fonction de moyenne 
$$
m(t) = \E(X_t) = 0
$$ 
et de fonction de covariance donnée par
\begin{eqnarray*}
C(s,t) &=& \Cov(X_s, X_t)\\
&=&\Cov(\frac{1}{\sqrt{a}}B_{as}, \frac{1}{\sqrt{a}}B_{at})\\
&=&\frac{1}{a}\Cov(B_{as}, B_{at})\\
&=&\frac{1}{a}as\land at\\
&=& s\land t.
\end{eqnarray*}
\end{solution}
\part Calculer $\E(X_7|X_5, X_3).$
\begin{solution}
$\E(X_7|X_5, X_3) = \E(X_7|X_5) = \E(X_7- X_5 + X_5|X_5) = X_5$
\end{solution}
\part Calculer $\E(X_7X_5X_3)$
\begin{solution}
$\E(X_7X_5X_3) = $
\end{solution}
\part Montrer que 
$$
M_t =  \exp(\lambda X_t - \frac{\lambda^2}{2}t),\text{ }t\geq 0.
$$
est une martingale. 
\begin{solution}
\end{solution}
\end{parts}

\end{questions}
% \bibliography{../lecture_notes/calcul_sto}
% \bibliographystyle{plain}
\end{document}
