\documentclass[11pt]{exam}
\usepackage[utf8]{inputenc}
\usepackage[T1]{fontenc}
\usepackage{amsmath, amssymb, amsopn, color, tikz, mathtools}
\usepackage[margin=1in]{geometry}
\usepackage{titlesec}
\usepackage{tipa}
\usepackage{hyperref}


% Style
\setlength\parindent{0pt}
\shadedsolutions

% Define course info
\def\semester{Semestre 2}
\def\course{Calcul stochastique M1 DUAS}
\def\name{P.-O. Goffard}
%\def\quizdate{10/5, 10/6}
\def\hwknum{}
%\def\title{\MakeUppercase{Homework \hwknum -- quiz \quizdate }}
\def\title{\MakeUppercase{TD 5: Intégrale stochastique}}
% Distributions.
\newcommand*{\UnifDist}{\mathsf{Unif}}
\newcommand*{\ExpDist}{\mathsf{Exp}}
\newcommand*{\DepExpDist}{\mathsf{DepExp}}
\newcommand*{\GammaDist}{\mathsf{Gamma}}
\newcommand*{\LognormalDist}{\mathsf{LogNorm}}
\newcommand*{\WeibullDist}{\mathsf{Weib}}
\newcommand*{\ParetoDist}{\mathsf{Par}}
\newcommand*{\NormalDist}{\mathsf{Normal}}

\newcommand*{\GeometricDist}{\mathsf{Geom}}
\newcommand*{\NegBinomialDist}{\mathsf{NegBin}}
\newcommand*{\BinomialDist}{\mathsf{Bin}}
\newcommand*{\PoissonDist}{\mathsf{Poisson}}
\newcommand*{\Cov}{\mathsf{Cov}}

% Sets of numbers.
\newcommand*{\RL}{\mathbb{R}}
\newcommand*{\N}{\mathbb{N}}
\newcommand*{\NZ}{\mathbb{N}_0}
% \newcommand*{\NL}{\mathbb{N}_+}

\newcommand*{\cond}{\mid}
\newcommand*{\given}{\,;\,}

%Probability symbols
\newcommand*{\Prob}{\mathbb{P}}
\newcommand*{\Q}{\mathbb{Q}}
\newcommand*{\E}{\mathbb{E}}
\newcommand*{\F}{\mathcal{F}}
% Regarding spacing and abbreviations.
\usepackage{xspace}

% Acronyms
% \@\xspace doesn't add space if next char is punctuation
% However, these will give 2 .'s if used at end of sentence.
\newcommand*{\eg}{e.g.\@\xspace}
\newcommand*{\ps}{p.s.\@\xspace}
\newcommand*{\ie}{i.e.\@\xspace}
\newcommand*{\va}{v.a.\@\xspace}
\newcommand*{\iid}{i.i.d.\@\xspace}
\newcommand*{\ssi}{s.s.i.\@\xspace}
\newcommand*{\cf}{c.f.\@\xspace}
\newcommand*{\pdf}{p.d.f.\@\xspace}
\newcommand*{\pmf}{p.m.f.\@\xspace}
\newcommand*{\cdf}{c.d.f.\@\xspace}
\newcommand*{\SMC}{\textbf{SMC}\@\xspace}
\newcommand*{\MCMC}{\textbf{MCMC}\@\xspace}
\newcommand*{\VF}{\textbf{VF}\@\xspace}


\newcommand*{\iidSim}{\overset{\mathrm{i.i.d.}}{\sim}}
\newcommand*{\bt}{\bm{\theta}}
\newcommand*{\bTheta}{\bm{\Theta}}
\newcommand*{\bbeta}{\bm{\beta}}
\newcommand*{\bx}{\mathbf{x}}
\newcommand*{\bs}{\bm{s}}
\newcommand*{\bu}{\bm{u}}
\newcommand*{\bn}{\bm{n}}

% Roman versions of things.
\newcommand*{\dd}{\mathop{}\!\mathrm{d}}
\newcommand*{\e}{\mathrm{e}}
\DeclareMathOperator*{\argmax}{arg\,max}
\DeclareMathOperator*{\argmin}{arg\,min}

\newcommand*{\norm}[1]{\lVert{} #1\rVert}

% \DeclarePairedDelimiterXPP{\ind}[1]{\ind_}{\{}{\}}{}{#1}
\newcommand*{\ind}{\mathbb{I}}
\def\euro{\mbox{\raisebox{.25ex}{{\it =}}\hspace{-.5em}{\sf C}}}
  \everymath{\displaystyle}
% \newcommand{\limsup}{\overline{\lim}\,}            % blackboard P
% \newcommand{\liminf}{\underline{\lim}\,}            % blackboard P

\begin{document}

% Heading
{\center \textsc{\Large\title}\\
	\vspace*{1em}
	\course -- \semester\\
	\name\\
	\vspace*{2em}
	\hrule
\vspace*{2em}}
\begin{questions}
\question Montrer que pour $f$ fonction déterministe, de carré intégrable, 
$$
\E\left(B_t\int_0^\infty f(s)\text{d}B_s\right) = \int_0^tf(s)\text{d}s
$$
\begin{solution}
On a 
\begin{eqnarray*}
\E\left(B_t\int_0^\infty f(s)\text{d}B_s\right) &=&\E\left(B_t\int_0^t f(s)\text{d}B_s + B_t\int_t^\infty f(s)\text{d}B_s\right)\\
&=&\E\left(\int_0^t \text{d}B_s\int_0^t f(s)\text{d}B_s\right) + \E(B_t)\E\left(\int_t^\infty f(s)\text{d}B_s\right)\\
&=&\int_0^t f(s)\text{d}s + 0
\end{eqnarray*}
\end{solution}

\question Soit $(X_t)_{t\geq0}$ un processus dont la dynamique est donnée par 
$$
\text{d}X_t = (1-2X_t)\text{d}t+3\text{d}B_t.
$$
\begin{parts}
\part Utiliser la formule d'Ito pour trouver $\text{d}(\e^{rt}X_t)$
\begin{solution}
On applique la formule d'Ito sur la fonction $f(t,X_t) = \e^{rt}X_t$, on a 
$$
\frac{\partial }{\partial t}f = r\e^{rt}X_t \text{ ; }\frac{\partial }{\partial x}f = \e^{rt}\text{ ; }\frac{\partial^2 f}{\partial x^2} = 0. 
$$
On en déduit que 
\begin{eqnarray*}
\text{d}(\e^{rt}X_t) &=& \left[ r\e^{rt}X_t + \e^{rt}(1-2X_t) + \frac{1}{2}\cdot 3^2\cdot0\right]\text{d}t + 3\e^{rt}\text{d}B_t =\text{d}(\e^{rt}X_t)\\ 
&=& \left[ r\e^{rt}X_t + \e^{rt}(1-2X_t) \right]\text{d}t + 3\e^{rt}\text{d}B_t 
\end{eqnarray*}
\end{solution}
\part Choisissez $r$ de façon à simplfier l'expression l'expression du drift (devant $\text{d}t$). Calculer la moyenne puis la moyenne asymptotique ($t\rightarrow \infty$) du processus.
\begin{solution}
On choisit $r = 2$, pour obtenir
$$
\text{d}(\e^{2t}X_t) = \e^{2t}\text{d}t + 3\e^{2t}\text{d}B_t.
$$
En intégrant entre $0$ et $t$, il vient 
$$
\e^{2t}X_t - X_0 = \frac{1}{2}(\e^{2t} - 1) + 3\int_0^t\e^{2s}\text{d}B_s.
$$

on en déduit que 
$$
\E(X_t) = \E(X_0)e^{-2t} + \frac{1}{2}(1 - \e^{-2t}) \rightarrow 1/2
$$
\end{solution}
\end{parts}
\question Soit un processus $X$ de dynamique 
$$
\text{d}X_t = \text{d}t + 2 \sqrt{X_t}\text{d}B_t, \text{ avec }X_0 = x_0.
$$
Montrer que le processus $X_t = (B_t + x_0)^2$ est une solution.
\begin{solution}
On applique la formule d'Ito sur la fonction $f(t, B_t) = (B_t + x_0)$. On a 
$$
\frac{\partial }{\partial t}f = 0 \text{ ; }\frac{\partial }{\partial x}f = 2(B_t + x_0)\text{ ; }\frac{\partial^2 f}{\partial x^2} = 2. 
$$
On en déduit que 
$$
\text{d}X_t = \left[0 + 0 \cdot 2(B_t + x_0) + \frac{1}{2}\cdot 2  \right]\text{d}t  + 2(B_t + x_0) \text{d}B_t = \text{d}t + + 2\sqrt{X_t}\text{d}B_t.
$$
\end{solution}
\question Soit le processus $X$ de dynamique 
$$
\begin{cases}
\text{d}X_t = \frac{X_t}{t-1}\text{d}t+\text{d}B_t,&\text{ }0\leq t<1,\\
X_0 = 0.&
\end{cases}
$$
\begin{parts}
\part Montrer que 
$$
X_t = (1-t)\int_0^t\frac{\text{d}B_s}{1-s}
$$
\begin{solution}
On applique la formule d'Ito à $f(t, X_t) = X_t/(1-t)$ pour obtenir
$$
\text{d}\left(\frac{X_t}{1-t}\right) =\frac{\text{d}B_t}{1-t}.
$$
Le résultat est obtenue par intégration entre $0$ et $t$. 
\end{solution}
\part Montrer que $X$ est un processus gaussien. Calculer sa fonction espérance et covariance
\begin{solution}
$X$ est une intégrale de Wiener, il s'agit bien d'un processus gaussien de fonction de moyenne identiquement nulle et de fonction de covariance donner par 
\begin{eqnarray*}
C(s,t) &=& s\land t\frac{(1-t)(1-s)}{1-s\land t} \\
&=&\begin{cases}
s(1-t),&\text{ si } s<t\\
t(1-s),&\text{ sinon}
\end{cases}
\end{eqnarray*}
Notre ami le pont brownien fait son come back.
\end{solution}
\part Montrer que $\underset{t\rightarrow 1}{\lim} X_t =0$ dans $\mathcal{L}^2$
\begin{solution}
On a 
$$
\mathbb{V}(X_t) = t(1-t)\rightarrow 0\text{ lorsque }t\rightarrow 1.
$$
\end{solution}
\end{parts}
\question Soit $X$ un processus de dynamique 
$$
\text{d}X_t = X_t\mu\text{d}t + X_t\sigma\text{d}B_t
$$
\begin{parts}
\part Calculer
$$
\E\left[(X_T-K)_+\right]
$$
Il s'agit du \textit{pay-off} espéré d'un call européen de maturité $T>0$ et de prix d'exercice $K>0$ dérivé l'actif $X$. Donner la réponse en fonction de $X_0,K,T,\mu,\sigma,\phi$ (dfr de la loi normale centrée-réduite) et 
$$
d_1 = \left[\ln(K/X_0)-\left(\mu - \frac{\sigma^2}{2}\right)T\right]\frac{1}{\sigma\sqrt{T}}.
$$
\begin{solution}
On obtient après application de la formule d'Ito sur $f(t,X_t)=\ln(X_t)$
$$
X_t = X_0\exp\left[\left(\mu - \frac{\sigma^2}{2}\right) + \sigma B_t\right],\text{ }t\geq 0.
$$
Le flux espéré du call est donnée par 
\begin{eqnarray*}
\E[(X_T-K)_+] &= &\E(X_T\ind_{X_T>K})- K\Prob(X_T>K)\\
&= &\E\left(X_0\exp\left[\left(\mu - \frac{\sigma^2}{2}\right)T + \sigma B_T\right]\ind_{X_0\exp\left[\left(\mu - \frac{\sigma^2}{2}\right)T + \sigma B_T\right]>K}\right)\\
&-& K\Prob\left(X_0\exp\left[\left(\mu - \frac{\sigma^2}{2}\right)T + \sigma B_T\right]>K\right)\\
&= &\E\left(X_0\exp\left[\left(\mu - \frac{\sigma^2}{2}\right)T + \sigma B_T\right]\ind_{X_0\exp\left[\left(\mu - \frac{\sigma^2}{2}\right)T + \sigma B_T\right]>K}\right)\\
&-& K\Prob\left(X_0\exp\left[\left(\mu - \frac{\sigma^2}{2}\right)T + \sigma B_T\right]>K\right)\\
&= &X_0\exp\left[\left(\mu - \frac{\sigma^2}{2}\right)T\right]\E\left( \e^{\sigma B_T}\ind_{\sigma B_T>\sigma \sqrt{T}d_1}\right)\\
&-& K\left[1-\phi(d_1)\right]
\end{eqnarray*}
où $d_1 = \left[\ln(K/X_0)-\left(\mu - \frac{\sigma^2}{2}\right)T\right]\frac{1}{\sigma\sqrt{T}}$
On a 
\begin{eqnarray*}
\E\left(\e^{\sigma B_T}\ind_{\sigma B_T>\sigma \sqrt{T}d_1}\right)&=&\int_{\sigma \sqrt{T}d_1}^{\infty}\e^{x}\frac{1}{\sqrt{2\pi}\sqrt{T}\sigma}\e^{-x^2 / 2T\sigma^2}\text{d}x\\
&=&\e^{\sigma^2 T / 2}\int_{\sigma \sqrt{T}d_1}^{\infty}\frac{1}{\sqrt{2\pi}\sqrt{T}\sigma}\e^{-\frac{1}{2}\left(\frac{x}{\sigma\sqrt{T}}  - \sigma\sqrt{T}\right)^2}\text{d}x\\
&=&\e^{\sigma^2 T / 2}\int_{d_1 - \sigma\sqrt{T}}^{\infty}\frac{1}{\sqrt{2\pi}}\e^{-\frac{z^2}{2}} \text{d}x\\
&=&\e^{\sigma^2 T / 2}\left(1 - \phi(d_1 - \sigma \sqrt{T})\right).
\end{eqnarray*}
On obtient finalement
$$
\E[(X_T-K)_+] = X_0\e^{\mu T}\left[1-\phi(d_1 - \sigma\sqrt{T})\right] - K\left[1-\phi(d_1)\right]
$$
\end{solution}
\part Montrer que $Y_t = \e^{-\mu t }X_t,\text{ }t\geq 0$ est une martingale.\\
% \underline{Indication:} Il faut appliquer deux fois la formule d'Ito.
\begin{solution}
On applique la formule d'Ito sur $Y_t = \e^{-\mu t }X_t = f(t, X_t)$. Il vient 
$$
\text{d}Y_t = \sigma Y_t\text{d}B_t.
$$
On en déduit que 
$$
Y_t = Y_0\e^{\sigma B_t -\frac{\sigma^2}{2}t}
$$
qui est une martingale.
\end{solution}

\end{parts}

\end{questions}

% \bibliography{../lecture_notes/calcul_sto}
% \bibliographystyle{plain}
\end{document}
