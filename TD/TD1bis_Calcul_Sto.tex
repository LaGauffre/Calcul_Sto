\documentclass[11pt, answers]{exam}
\usepackage[utf8]{inputenc}
\usepackage[T1]{fontenc}
\usepackage{amsmath, amssymb, amsopn, color, tikz, mathtools}
\usepackage[margin=1in]{geometry}
\usepackage{titlesec}
\usepackage{tipa}
\usepackage{hyperref}


% Style
\setlength\parindent{0pt}
\shadedsolutions

% Define course info
\def\semester{Semestre 2}
\def\course{Calcul stochastique M1 DUAS}
\def\name{P.-O. Goffard}
%\def\quizdate{10/5, 10/6}
\def\hwknum{}
%\def\title{\MakeUppercase{Homework \hwknum -- quiz \quizdate }}
\def\title{\MakeUppercase{TD 1: Marche aléatoire et martingale en temps discret.}}
% Distributions.
\newcommand*{\UnifDist}{\mathsf{Unif}}
\newcommand*{\ExpDist}{\mathsf{Exp}}
\newcommand*{\DepExpDist}{\mathsf{DepExp}}
\newcommand*{\GammaDist}{\mathsf{Gamma}}
\newcommand*{\LognormalDist}{\mathsf{LogNorm}}
\newcommand*{\WeibullDist}{\mathsf{Weib}}
\newcommand*{\ParetoDist}{\mathsf{Par}}
\newcommand*{\NormalDist}{\mathsf{Normal}}

\newcommand*{\GeometricDist}{\mathsf{Geom}}
\newcommand*{\NegBinomialDist}{\mathsf{NegBin}}
\newcommand*{\BinomialDist}{\mathsf{Bin}}
\newcommand*{\PoissonDist}{\mathsf{Poisson}}

% Sets of numbers.
\newcommand*{\RL}{\mathbb{R}}
\newcommand*{\N}{\mathbb{N}}
\newcommand*{\NZ}{\mathbb{N}_0}
% \newcommand*{\NL}{\mathbb{N}_+}

\newcommand*{\cond}{\mid}
\newcommand*{\given}{\,;\,}

%Probability symbols
\newcommand*{\Prob}{\mathbb{P}}
\newcommand*{\Q}{\mathbb{Q}}
\newcommand*{\E}{\mathbb{E}}
\newcommand*{\F}{\mathcal{F}}
% Regarding spacing and abbreviations.
\usepackage{xspace}

% Acronyms
% \@\xspace doesn't add space if next char is punctuation
% However, these will give 2 .'s if used at end of sentence.
\newcommand*{\eg}{e.g.\@\xspace}
\newcommand*{\ps}{p.s.\@\xspace}
\newcommand*{\ie}{i.e.\@\xspace}
\newcommand*{\va}{v.a.\@\xspace}
\newcommand*{\iid}{i.i.d.\@\xspace}
\newcommand*{\ssi}{s.s.i.\@\xspace}
\newcommand*{\cf}{c.f.\@\xspace}
\newcommand*{\pdf}{p.d.f.\@\xspace}
\newcommand*{\pmf}{p.m.f.\@\xspace}
\newcommand*{\cdf}{c.d.f.\@\xspace}
\newcommand*{\SMC}{\textbf{SMC}\@\xspace}
\newcommand*{\MCMC}{\textbf{MCMC}\@\xspace}
\newcommand*{\VF}{\textbf{VF}\@\xspace}


\newcommand*{\iidSim}{\overset{\mathrm{i.i.d.}}{\sim}}
\newcommand*{\bt}{\bm{\theta}}
\newcommand*{\bTheta}{\bm{\Theta}}
\newcommand*{\bbeta}{\bm{\beta}}
\newcommand*{\bx}{\mathbf{x}}
\newcommand*{\bs}{\bm{s}}
\newcommand*{\bu}{\bm{u}}
\newcommand*{\bn}{\bm{n}}

% Roman versions of things.
\newcommand*{\dd}{\mathop{}\!\mathrm{d}}
\newcommand*{\e}{\mathrm{e}}
\DeclareMathOperator*{\argmax}{arg\,max}
\DeclareMathOperator*{\argmin}{arg\,min}

\newcommand*{\norm}[1]{\lVert{} #1\rVert}

% \DeclarePairedDelimiterXPP{\ind}[1]{\ind_}{\{}{\}}{}{#1}
\newcommand*{\ind}{\mathbb{I}}
\def\euro{\mbox{\raisebox{.25ex}{{\it =}}\hspace{-.5em}{\sf C}}}
  \everymath{\displaystyle}
% \newcommand{\limsup}{\overline{\lim}\,}            % blackboard P
% \newcommand{\liminf}{\underline{\lim}\,}            % blackboard P

\begin{document}

% Heading
{\center \textsc{\Large\title}\\
	\vspace*{1em}
	\course -- \semester\\
	\name\\
	\vspace*{2em}
	\hrule
\vspace*{2em}}

\begin{questions}

\question Une urne contient $N$ boules, dont $0<a<N$ boules blanches et $b = N-a$ boules noires. A chaque pas de temps, on tire une boule au hasard dans l'urne. 
\begin{itemize}
\item Si la boule est blanche on remet la boule blanche et on rajoute une boule blanche. 
\item Si cette boule est noire, on replace la boule noire dans l'urne eton rajoute une boule noire. 
\end{itemize}
On répète ce procédé et on s'intéresse aux processus $(Y_n)_{n\geq 0}$ et $(X_{n})_{n\geq 0 }$ qui correspondent respectivement aux nombres et à la proportion de boules blanches dans l'urne après $n$ tirage.
\begin{parts}
\part Exprimer $X_n$ en fonction de $Y_n$.
\begin{solution}
$X_n = \frac{Y_n}{N + n}$
\end{solution}
\part Soit $\F_n = \sigma(Y_1, \ldots, Y_n)$, calculer $\Prob(Y_{n+1} = Y_n+1|\F_n)$ et $\Prob(Y_{n+1} = Y_n|\F_n)$.
\begin{solution}
$\Prob(Y_{n+1} = Y_n+1|\F_n) = \frac{Y_n}{N+n}$ et $\Prob(Y_{n+1} = Y_n|\F_n) = \frac{N+n - Y_n}{N+n}$ 
\end{solution}
\part Montrer que $(X_n)_{n\geq0}$ est une $\F_n$-martingale. Quel conclusion intéressante au sujet de la proportion moyenne de boule blanche dans l'urne peut-on tirer de cela?
\begin{solution}
\begin{eqnarray*}
\E(X_{n+1}|\F_n) &=& \E\left[\frac{Y_{n+1}}{N+n+1}|\F_n\right]\\
&=& \frac{1}{N+n+1}\left[(Y_n + 1)\frac{Y_{n}}{N+n}+Y_n \frac{N+n - Y_{n}}{N+n}\right]\\
&=&\frac{Y_n}{N+n} = X_n
\end{eqnarray*}
\end{solution}
\part Supposons que $a=b = 1$, montrer par récurrence sur $n\geq0$ que $Y_n\sim\UnifDist(\{1, \ldots, n+1\})$ 
\begin{solution}
Si $a=b=1$ alors $N = 2$. On raisonne par récurrence, $Y_0=1\sim\UnifDist(\{1\})$, la propriété est vérifiée au rang $n=0$. \\

\noindent Supposons la propriété vraie au rang $n$, qu'en est il au rang $n+1$?\\

Par la formules des probabilités totale, on a pour $k\geq 2$,
\begin{eqnarray*}
\Prob(Y_{n+1} = k)&=&\sum_{l=1}^{n+1}\Prob(Y_{n+1} = k|Y_{n} = l)\frac{1}{n+1}\\
&=&\Prob(Y_{n+1} = k|Y_{n} = k)\frac{1}{n+1}+\Prob(Y_{n+1} = k|Y_{n} = k-1)\frac{1}{n+1}\\
&=&\frac{2+n-k}{2+n}\frac{1}{n+1}+\frac{k-1}{2+n}\frac{1}{n+1}\\
&=&\frac{1}{n+2}
\end{eqnarray*}
Pour $k=1$, on a 
\begin{eqnarray*}
\Prob(Y_{n+1} = 1)&=&\sum_{l=1}^{n+1}\Prob(Y_{n+1} = 1|Y_{n} = l)\frac{1}{n+1}\\
&=&\Prob(Y_{n+1} = 1|Y_{n} = 1)\frac{1}{n+1}\\
&=&\frac{2+n-1}{2+n}\frac{1}{n+1}\\
&=&\frac{1}{n+2}
\end{eqnarray*}
La propriété est ainsi vérifiée au rang $n+1$.
\end{solution}
\part En déduire la loi de $X_{n}$ et de $X_\infty$ après en avoir justifié l'existence.\\
\underline{Indication:} Pour la loi de $X_\infty$, il faut étudié la limite de $\E(e^{\theta X_n})$ lorsque $n\rightarrow \infty$.
\begin{solution}
On a 
$$
\Prob\left(X_n =\frac{k}{n+2}\right) =\frac{1}{n+1},\text{ pour }k = 1,\ldots, n+1. 
$$
Comme $(X_n)_{n\geq0}$ est une martingale positive et bornée (par 1) alors elle converge presque sûrement.
On a 
$$
\E(\e^{\theta X_n})=\frac{1}{n+1}\e^{\theta/(n+1)}\frac{1-\e^\theta}{1-\e^{\theta/(n+1)}}\rightarrow \frac{\e^\theta - 1}{\theta}\text{ lorsque }n\rightarrow\infty
.$$
C'est la loi uniforme (continue) sur $(0,1)$.
\end{solution}
\end{parts}
\question Le processus de branchement permet de suivre l'évolution d'une population. Soit $X$ une variable aléatoire de comptage telle que
$$
\Prob(X = 0)>0\text{ et }\E(X)<\infty.
$$
$X$ correspond au nombre d'enfants d'un individu. On définit une suite (à deux indices) de variables aléatoires
$$
\left(X^{(n)}_r\right)_{n,r\in \N}
$$ 
indépendantes et distribuées comme $X$ de sorte que $X^{(n+1)}_r$ désigne le nombre de descendants (membre de la génération $n+1$) de l'individu $r$ (qui appartient à la génération $n$). Le processus $(Z_n)_{n\in \N}$ définie par 
$$
Z_0 =1\text{, }Z_{n} = X^{(n)}_1+\ldots+ X^{(n)}_{Z_{n-1}} 
$$
correspond à la taille de la génération $n$.
\begin{parts}
\part On note $\mu = \E(X)$, le nombre d'enfant moyen par individu. Montrer que 
$$
\E(Z_{n}) = \mu^n.
$$
En déduire la limite de $\E(Z_n)$ en fonction des valeur de $\mu$.
\begin{solution}
On a 
$$
\E(Z_{n}) = \E(\sum_i^{Z_{n-1}}X_i^{(n)})=\E[\E(\sum_i^{Z_{n-1}}X_i^{(n)}|Z_{n-1})] =\E(Z_{n-1})\mu = \ldots =  \E(Z_{0})\mu^n= \mu^n
$$
On en déduit que 
$$
\begin{cases}
\E(Z_{n})\rightarrow \infty&\text{ si }\mu>1\\
\E(Z_{n})\rightarrow 1&\text{ si }\mu=1\\
\E(Z_{n})\rightarrow 0&\text{ si }\mu<1\\
\end{cases}
$$
\end{solution}
\part Montrer que 
$$
M_n = Z_n /\mu^n, n\geq 0
$$
est une martingale.
\begin{solution}
$$\E(M_n|\F_{n-1})=\E(Z_n|\F_{n-1})/\mu^n= Z_{n-1}\mu/\mu^n = M_{n-1}.$$
\end{solution}
\part Soit 
$$
G_X(\theta) = \E(\theta^X),
$$
la fonction génératrice des probabilités de $X$. Montrer que 
$$
G_{Z_n}(\theta) = G_{Z_{n-1}}[G_{X}(\theta)]
$$
puis que c'est équivalent à 
$$
G_{Z_n}(\theta) = G_{X}[G_{Z_{n-1}}(\theta)].
$$
\begin{solution}
On a 
\begin{eqnarray*}
G_{Z_n}(\theta) &=&\E(\theta^{Z_n})\\
 &=&\E(\theta^{\sum_i^{Z_{n-1}}X_i^{(n)}})\\
 &=&\E[\E(\theta^{\sum_i^{Z_{n-1}}X_i^{(n)}}|Z_{n-1})]\\
 &=&\E[\prod_i^{Z_{n-1}}\E(\theta^{X_i^{(n)}})]\\
 &=&\E[\prod_i^{Z_{n-1}}\E(\theta^{X})]\\
 &=&\E[\E(\theta^{X})^{Z_{n-1}}]\\
 &=&\E[\E(\theta^{X})^{Z_{n-1}}]\\
 &=&G_{Z_{n-1}}[G_{X}(\theta)].\\
\end{eqnarray*}
On en déduit 
$$
G_{Z_n}(\theta) = G_{Z_{n-1}}[G_{X}(\theta)] = G_X\circ G_X\circ\ldots G_X(\theta)=  G_{X}[G_{Z_{n-1}}(\theta)]. 
$$
\end{solution}
\part Soit 
$$
\pi_n = \Prob(Z_n = 0),\text{ }n\geq1,
$$
la probabilité que la population soit éteinte à la génération $n$. Quel est le lien entre $\pi_n$ et $G_{Z_n}(\theta)$? 
\begin{solution}
$\pi_n = G_{Z_n}(0)$
\end{solution}
\part En déduire que 
$$
\pi_n = G_X(\pi_{n-1}).
$$
\begin{solution}
On évalue $G_{Z_n}(\theta) = G_{X}[G_{Z_{n-1}}(\theta)]$ en $\theta = 0$.
\end{solution}
\part Montrer que $\pi_n$ est une suite convergente. Quelle est l'interprétation de $\pi=\lim \pi_n$?
\begin{solution}
On a $\{Z_{n-1}=0\}\subset\{Z_{n}=0\}$ donc $\pi_n\geq\pi_{n-1}$. $(\pi_n)$ est une suite croissante et bornée (par $1$), donc elle converge. Sa limite $\pi$ correspond à la probabilité que la population s'éteigne un un jour. $1-\pi$ est la probabilité que la population ne s'éteigne jamais.
\end{solution}
\part Quel est le lien entre $\mu=\E(X)$ et $G_X(\theta)$?
\begin{solution}
$\mu = G_X'(1)$
\end{solution}
\part Déduire des questions c),d) et f) que 
$$
\begin{cases}
0< \pi<1,& \text{ si }\mu <1,\\
\pi=1.& \text{ si }\mu \geq 1.
\end{cases}
$$
\underline{Indication:} le raisonnement s'appuie sur une étude de la fonction $G_X$ sur $[0,1]$.
\begin{solution}
On observe que $\pi$ est solution de 
$$
\pi = G_X(\pi),
$$
en passant 
$$
\pi_n = G_X(\pi_{n-1}),
$$
à la limite. On note que $G_X(0)=\Prob(X=0)\in(0,1)$ et $G_X(1)=1$. De plus, $G_X'(\theta)\geq0$ et $G_X'(\theta)\geq0$  pour $\theta \in[0,1]$. Soit $f(\theta)=G_X(\theta)-\theta$, $f'(\theta)=G_X'(\theta)-1$ et $f'(\theta)=G_X''(\theta)$. On en déduit que $f'(\theta)\in[-1, \mu-1]$. 
\begin{itemize}
\item Si $\mu \leq 1$ alors $f$ est strictement croissante sur $[0, 1]$ et $f(\pi)=0$ n'a pas de solution $<1$. 
\item Si $\mu>1$, il existe $\theta^\ast$ tel que $f(\theta^\ast)<0$ et donc $\pi\in[0,\theta^\ast]$ tel que $f(\pi)=0$.
\end{itemize}
\end{solution}

\end{parts}
\end{questions}
\end{document}
